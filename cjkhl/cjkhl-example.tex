

% Derived from an answer to the question
% http://tex.stackexchange.com/questions/75019/is-there-any-resolution-for-highlighting-text-in-cjk

\documentclass[nofonts]{ctexbook}
\setCJKmainfont{SimSun}

\definecolor{lightblue}{rgb}{.8,.8,1}


\usepackage{cjkhl}

\renewcommand\cjkhlbleed{.07pt}

\begin{document}

本条所说的三日期间与第98条规定的三日期间在计算上采同样的规则(请参考第98条的释义部分),期间的开始均不是以扣押命令的发出为标准,而是以扣押的生效为准。\cjkhl{yellow}{对邮件的扣押,以邮政服务机构收到扣押命令,邮 ; 件开始处于被截留的状态,视为扣押的生效(Pfeiffer S471 Rn4)},期间由此开始。期间的开始计算同样适用第42条的规定。
但与第98条不同的是,第98条的三日期间只是向法官提出追认照准的申请的期间,而本条则是收到法官追认照准的期间。邮政服务机构如果在三日内没收到法官的追认,检察官的扣押命令失效,毋需再向检察官交出邮件。但邮件如果已经被交出的,则暂时不被返还,仍可保留在检察官处。如果在三日期满后法官又追认照准的,视为法官作出了新的扣押命令(Meyer-Goßner S325 Rn7)。
\cjkhl{lightblue}{如果法官在三日内不予照准的呢?如果在三日内没有追认,在三日后也没有作出追认的?如果法官在三日后作出不予照准的呢?六种情况}


\bigskip
\hrule
\bigskip
\obeylines



本条所说的三日期间与第98条规定的三日期间在计算上采同样的规则
(请参考第98条的释义部分),期间的开始均不是以扣押命令的发出为标准,而是以扣押的生效为准。\cjkhl{yellow}{对邮件的扣押,
以邮政服务机构收到扣押命令,邮 ; 件开始处于被截留的状态,
视为扣押的生效(Pfeiffer S471 Rn4)},期间由此开始。期间的开始计算同样适用第42条的规定。
但与第98条不同的是,第98条的三日期间只是向法官提出追
认照准的申请的期间,而本条则是收到法官追认照
准
的期间。邮政服务机构如果在三日内没收到法官的追认,检察官的扣押命令失效,毋需再向检察官交出邮件。但邮件如果已经被交出的,则暂时不被返还,仍可保留在检察官处。如果在三日期满后法官又追认照准的,视为法官作出
了新的扣押命令(Meyer-Goßner S325 Rn7)。
\cjkhl{lightblue}{如果法官在三日内不予照准的呢?如
果在三日内没有追认,在三日后也没有作出追认的?
如果法官在三日后作出不予照准的呢?六种情况}
\end{document}
