\documentclass[a4paper]{article}

% Program: scriptfonts.tex
% Copyright 2000 Robin Fairbairns
%
% This program can redistributed and/or modified under the terms
% of the LaTeX Project Public License Distributed from CTAN
% archives as file macros/latex/base/lppl.txt (the file is also
% available from http://www.latex-project.org); either version 1.2 of
% the License, or (at your option) any later version.

%% this is a bit of a tricksy document: don't be surprised if it
%% doesn't behave properly when you compile it, either with latex
%% (pkgindoc _isn't_ part of the standard distribution) or with
%% pdflatex (additionally, as originally configured, my pdftex didn't
%% download the zapf chancery font so that the output printed as an
%% "ordinary" sort of font, which rather spoiled the effect)

\usepackage{mathrsfs}
\usepackage{graphics}

\usepackage{pkgindoc} % allow \usepackage commands in body of document
                      % (may be generated from latex sources)

% fudges to allow us to run eucal.sty (see more detailed comment just
% before \includegraphics{eucal}
\DeclareMathAlphabet\EuScript{U}{eus}{m}{n}
\SetMathAlphabet\EuScript{bold}{U}{eus}{b}{n}

\DeclareRobustCommand\cs[1]{\texttt{\char`\\#1}}

\DeclareMathAlphabet{\mathpzc}{OT1}{pzc}{m}{it}

%\DeclareFontFamily{OT1}{pzcs}{}
%\DeclareFontShape{OT1}{pzcs}{m}{it}{<-> s * [0.900] pzcmi7t}{}
%\DeclareMathAlphabet{\mathpzcs}{OT1}{pzc}{m}{it}

\begin{document}
\title{Script fonts for mathematical use\\ available to \LaTeX{} users}
\author{Robin Fairbairns}
\date{November 2000}
\maketitle

This short sampler document is mostly taken from an answer provided to
me for the UK TUG FAQ by Anthony Goreham.

\section{Default script fonts}

The \cs{mathcal} command gives us:

$$\mathcal{ABCDEFGHIJKLMNOPQRSTUVWXYZ}$$

% the .eps files are generated with \Huge set; we measure T in that
% size for generating \raisebox parameters.
\newlength\depthT
\settoheight\depthT{\Huge$\mathcal{T}$}

\noindent\textbf{Caveat:} The AMS Postscript Type~1 versions of the Computer
Modern fonts (which are used for the Postscript and PDF distributions
of this document) are based on Knuth's `old' shape for the letter
``T'': compare the current (Meta\-font) version~---
\raisebox{-0.4\depthT}{\includegraphics{mathcalTpk}} with the Type
1 version~--- \raisebox{-0.4\depthT}{\includegraphics{mathcalTt1}}

\section{Euler script fonts}

The package \texttt{eucal} changes the definition of \cs{mathcal} to
give us:

% because eucal redeclares \mathcal, we can't actually execute the
% package until after we've done our first example.  however, the
% package uses nfss commands which are themselves only available in
% the preamble, so we execute those in the preamble, and dummy them
% out here
\renewcommand\DeclareMathAlphabet[5]{\ignorespaces}
\renewcommand\SetMathAlphabet[6]{\ignorespaces}
\usepackage{eucal}

$$\mathcal{ABCDEFGHIJKLMNOPQRSTUVWXYZ}$$

\section{Ralph Smith's Formal Script}

The package \texttt{mathrsfs} gives us a new command \cs{mathscr}, and
with that we get

$$\mathscr{ABCDEFGHIJKLMNOPQRSTUVWXYZ}$$

\section{Using the Adobe Zapf Chancery font}

There's no defined package for this rather pleasing font.  To get the
font at its default size, one may say (in the document preamble):
\begin{verbatim}
\DeclareMathAlphabet{\mathpzc}{OT1}{pzc}{m}{it}
\end{verbatim}
which declares a command \cs{mathpzc} which give us:

$$\mathpzc{ABCDEFGHIJKLMNOPQRSTUVWXYZ}$$

\noindent which (in Anthony Goreham's view) may seem a bit large.  He
gives a recipe for producing a smaller version, in the FAQ answer.

% this doesn't seem to have the desired effect.  omitted from version 1
%
%The sequence
%\begin{verbatim}
%\DeclareFontFamily{OT1}{pzc}{}
%\DeclareFontShape{OT1}{pzc}{m}{it}{<-> s * [0.900] pzcmi7t}{}
%\DeclareMathAlphabet{\mathpzc}{OT1}{pzc}{m}{it}
%\end{verbatim}
%gives us:
%
%$$\mathpzcs{ABCDEFGHIJKLMNOPQRSTUVWXYZ}$$

\end{document}
