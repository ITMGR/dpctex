% \iffalse
%% Source File: shellesc.dtx
%% Copyright 2015 David Carlisle
%%
%% This file may be distributed under the terms of the LPPL.
%% See README for details.
%
%<*dtx>
          \ProvidesFile{shellesc.dtx}
%</dtx>
%<package>\NeedsTeXFormat{LaTeX2e}
%<package>\ProvidesPackage{shellesc}
%<driver> \ProvidesFile{shellesc.drv}
% \fi
%         \ProvidesFile{shellesc.dtx}
       [2015/12/05 v0.01 unified shell escape interface for LaTeX (DPC)]
%
% \iffalse
%<*driver>
\documentclass{ltxdoc}
\begin{document}
\DocInput{shellesc.dtx}
\end{document}
%</driver>
% \fi
%
% \GetFileInfo{shellesc.dtx}
%
% \title{The \textsf{shellesc} Package\thanks{This file
%        has version number \fileversion, last
%        revised \filedate.
% Please report any issues at https://github.com/davidcarlisle/dpctex/issues}}
% \author{David Carlisle}
% \maketitle
%
% \section{Introduction}
%
% 
% \section{Implementation}
%
%    \begin{macrocode}
%<*package>
%    \end{macrocode}
%
% \subsection{Status Check}
%
%
%    \begin{macrocode}
\ifcase
  \ifx\pdfshellescape\@undefined
    \ifx\shellescape\@undefined
      \ifx\directlua\@undefined
        \z@
      \else
        \directlua{tex.sprint(os.execute() .. " ")}
      \fi
    \else
      \shellescape
    \fi
  \else
    \pdfshellescape
  \fi
  \PackageWarning{shellesc}{Shell escape disabled}
\or
  \PackageInfo   {shellesc}{Unrestricted shell escape ensabled}
\else
  \PackageInfo   {shellesc}{Restricted shell escape ensabled}
\fi
%    \end{macrocode}
%
% \subsection{The shellesc package interface}
%
% \begin{macro}{\ShellEscape}
% Execute the supplied tokens as a system dependent command, assuming 
% such execution is allowed.
%    \begin{macrocode}
\ifx\lastsavedimageresourcepages\@undefined
  \protected\def\ShellEscape{\immediate\write18 }
%    \end{macrocode}
%
%    \begin{macrocode}
\else
  \protected\def\ShellEscape#1{%
    \directlua{os.execute("\luaescapestring{#1}")}}
\fi
%    \end{macrocode}
% \end{macro}
%
% \begin{macro}{\DelayedShellEscape}
% Execute the supplied tokens as a system dependent command, when this
% node is shipped out with the completed page, assuming 
% such execution is allowed.
%    \begin{macrocode}
\ifx\lastsavedimageresourcepages\@undefined
  \protected\def\ShellEscape{\relax\write18 }
%    \end{macrocode}
%
%    \begin{macrocode}
\else
  \protected\def\DelayedShellEscape#1{%
    \latelua{os.execute("\luaescapestring{#1}")}}
\fi
%    \end{macrocode}
% \end{macro}
%
%
%
% \subsection{The write18 package interface}
%
% In web2c based engines other than Lua\TeX, |\write18| may be used
% directly.  The same was true in older LuaTeX, but from version 0.85
% onwards that is not available.
%
% The above |shellesc| package interface is recommended for new code,
% however for ease of porting existing documents and packages to newer
% Lua\TeX\ releases, a |\write18| interface is provided here via a
% call to Lua's |os.execute|.
%
% Note that as currently written this always does an \emph{immediate}
% call to the system.
%
% |\immediate| is supported but ignored, |\immediate\write18| and
% |\write18| both execute immediately. To use a delayed execution at
% the next shipout, use the |\DelayedShellEscape| command defined
% above.
%
% Note that it would be easy to make |\wriete18| defined here use
% delayed execution, just use |\DelayedShellEscape| instead of
% |ShellEscape| in the definition below. However detecting
% |\immediate| is tricky so the choice here is to always use the
% immediate form, which is overwhelmingly more commonly used with
% |\write18|.
%
% stop at this point if not a recent Lua\TeX.
%    \begin{macrocode}
\ifx\lastsavedimageresourcepages\@undefined\expandafter\endinput\fi
%    \end{macrocode}
%
%    \begin{macrocode}
\directlua{%
%    \end{macrocode}
%
%    \begin{macrocode}
shellesc = shellesc or {}
%    \end{macrocode}
%
% Lua function to use the token scanner to grab the following \TeX\
% number, and then test if stream 18 is being used, and then insert an
% appropriate \TeX\ command to handle the following brace group in
% each case.
%    \begin{macrocode}
local function write_or_execute()
  local s = token.scan_int()
  if (s==18) then
     tex.sprint(\the\numexpr\catcodetable@atletter\relax,
                "\string\\ShellEscape ")
  else
     tex.sprint(\the\numexpr\catcodetable@atletter\relax,
                "\string\\shellesc@write " .. s)
  end
end
%    \end{macrocode}
%
%    \begin{macrocode}
shellesc.write_or_execute=write_or_execute
%    \end{macrocode}
%
%    \begin{macrocode}
}
%    \end{macrocode}
%
%    \begin{macrocode}
\let\shellesc@write\write
%    \end{macrocode}
%
%    \begin{macrocode}
\protected\def\write{\directlua{shellesc.write_or_execute()}}
%    \end{macrocode}
%
%
%
%    \begin{macrocode}
%</package>
%    \end{macrocode}
%
% \Finale
%
