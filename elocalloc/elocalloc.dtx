% \iffalse
%% Source File: textcase.dtx
%% Copyright 2015 David Carlisle
%%
%% This file may be distributed under the terms of the LPPL.
%% See README for details.
%
%<*dtx>
          \ProvidesFile{elocalloc.dtx}
%</dtx>
%<package>\NeedsTeXFormat{LaTeX2e}
%<package>\ProvidesPackage{elocalloc}
%<driver> \ProvidesFile{elocalloc.drv}
% \fi
%         \ProvidesFile{elocalloc.dtx}
       [2015/05/09 v0.01 local allocation for LaTeX 2015+ (DPC)]
%
% \iffalse
%<*driver>
\documentclass{ltxdoc}
\begin{document}
\DocInput{elocalloc.dtx}
\end{document}
%</driver>
% \fi
%
% \GetFileInfo{elocalloc.dtx}
%
% \title{The \textsf{elocalloc} Package\thanks{This file
%        has version number \fileversion, last
%        revised \filedate.
% Please report any issues at https://github.com/davidcarlisle/dpctex/issues}}
% \author{David Carlisle}
% \maketitle
%
% \section{Introduction}
%
% Prior to the 2015/01/01 \LaTeX\ release, access to extended registers was
% usually provided by the \textsf{etex} package. From 2015 on the base
% \LaTeX\ format ``knows'' the register ranges for tex, etex, xetex and
% luatex, and so for most purposes the \textsf{etex} package is not
% required.
%
% For existing documents \textsf{etex} may still be loaded and will work
% as before, however in general it is best not to load \textsf{etex} as it
% will over-write the new allocation scheme with its own.
%
% Standard allocation macros such as |\newbox| and extended versions of
% tracing commands are now defined in the format for suitable engines however
% there were some features of \textsf{etex} not copied into the format,
% notably the ``local'' allocation macros. A search of CTAN showed that
% these were almost never used, and their use can often be avoided, however
% there are occasions when they may be useful. This package provides
% implementations based on the new allocation system in \LaTeX\ 2015/01/01.
%
% Within a local group a command such as |\locbox\tmpbox| defines |\tmpbox|
% to be a box register, but unlike |\newbox| the definition is local, and
% at the end of the group, |\tmpbox| loses its definition, and the box
% allocation number is restored. 
%
% The package defines: |\loccount|, |\locdimen|, |\locskip|, |\locmuskip|,
% |\locbox|, |\loctoks| and |\locmarks|. 
% 
% 
% \section{Implementation}
%
%    \begin{macrocode}
%<*package>
%    \end{macrocode}
% On old \LaTeX, load |etex.sty| and stop.
%    \begin{macrocode}
\ifx\e@alloc\@undefined
\RequirePackage{etex}
\expandafter\endinput
\fi
%    \end{macrocode}
%
% If something else defined |\locbox|, stop with a warning.
%
%    \begin{macrocode}
\ifx\locbox\@undefined\else
\PackageWarning{elocalloc}{%
   \string\locbox\space already defined, stopping}
\expandafter\endinput
\fi
%    \end{macrocode}
%
% Don't allow |\extrafloats| while a local allocation is in force.
% It is however OK to use |\extrafloats| with this package and to
% use floats within a group with a local allocation.
%
% For local allocations, reduce the top of the range available for global
% allocations by one, and locally allocate the register from that top
% position.
%
% Note that this means that (unlike the \textsf{etex} package originals)
% locally allocating one register affects the top of the other registers
% that share the same top of range, but doing otherwise would mean storing
% separate values ofr each type, and would make it harder to make
% |\extrafloats| work, should such an extension ever be needed.
%    \begin{macrocode}
\def\eloc@lloc#1#2#3#4#5{%
  \def\extrafloats##1{%
    \PackageWarning{elocalloc}{\string\extrafloats\space ignored}}
  \e@ch@ck{#1}#2\z@#3%
  \expandafter\e@alloc@chardef\expandafter#2%
                         \the\numexpr#2-1\relax
   \allocationnumber=#2%
  #3#5#2}
%    \end{macrocode}
%
% The top level local allocation commands.
%    \begin{macrocode}
\def\loccount {\eloc@lloc{\count 10}\float@count\countdef\count}
\def\locdimen {\eloc@lloc{\count 11}\float@count\dimendef\dimen}
\def\locskip  {\eloc@lloc{\count 12}\float@count\skipdef\skip}
\def\locmuskip{\eloc@lloc{\count 13}\e@alloc@top\muskipdef\muskip}
\def\locbox   {\eloc@lloc{\count 14}\float@count\e@alloc@chardef\box}
\def\loctoks  {\eloc@lloc{\count 15}\e@alloc@top\toksdef\toks}
\def\locmarks {\eloc@lloc{\count256}\e@alloc@topt\e@alloc@chardef\marks}
%    \end{macrocode}
%
%    \begin{macrocode}
%</package>
%    \end{macrocode}
%
% \Finale
%
